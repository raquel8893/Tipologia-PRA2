\documentclass[12pt]{article}
\usepackage{setspace}  
\usepackage{lmodern}
\usepackage{amssymb,amsmath}
\usepackage{ifxetex,ifluatex}
\usepackage{fixltx2e} % provides \textsubscript
\ifnum 0\ifxetex 1\fi\ifluatex 1\fi=0 % if pdftex
  \usepackage[T1]{fontenc}
  \usepackage[utf8]{inputenc}
\else % if luatex or xelatex
  \ifxetex
    \usepackage{mathspec}
  \else
    \usepackage{fontspec}
  \fi
  \defaultfontfeatures{Ligatures=TeX,Scale=MatchLowercase}
\fi
% use upquote if available, for straight quotes in verbatim environments
\IfFileExists{upquote.sty}{\usepackage{upquote}}{}
% use microtype if available
\IfFileExists{microtype.sty}{%
\usepackage{microtype}
\UseMicrotypeSet[protrusion]{basicmath} % disable protrusion for tt fonts
}{}
\usepackage[margin=1in]{geometry}
\usepackage{hyperref}
\hypersetup{unicode=true,
            pdftitle={Dataset: Accidentes de avión acontecidos a nivel mundial},
            pdfauthor={Teguayco Gutiérrez González},
            pdfborder={0 0 0},
            breaklinks=true}
\urlstyle{same}  % don't use monospace font for urls
\usepackage{graphicx,grffile}
\makeatletter
\def\maxwidth{\ifdim\Gin@nat@width>\linewidth\linewidth\else\Gin@nat@width\fi}
\def\maxheight{\ifdim\Gin@nat@height>\textheight\textheight\else\Gin@nat@height\fi}
\makeatother
% Scale images if necessary, so that they will not overflow the page
% margins by default, and it is still possible to overwrite the defaults
% using explicit options in \includegraphics[width, height, ...]{}
\setkeys{Gin}{width=\maxwidth,height=\maxheight,keepaspectratio}
\IfFileExists{parskip.sty}{%
\usepackage{parskip}
}{% else
\setlength{\parindent}{0pt}
\setlength{\parskip}{6pt plus 2pt minus 1pt}
}
\setlength{\emergencystretch}{3em}  % prevent overfull lines
\providecommand{\tightlist}{%
  \setlength{\itemsep}{0pt}\setlength{\parskip}{0pt}}
\setcounter{secnumdepth}{0}
% Redefines (sub)paragraphs to behave more like sections
\ifx\paragraph\undefined\else
\let\oldparagraph\paragraph
\renewcommand{\paragraph}[1]{\oldparagraph{#1}\mbox{}}
\fi
\ifx\subparagraph\undefined\else
\let\oldsubparagraph\subparagraph
\renewcommand{\subparagraph}[1]{\oldsubparagraph{#1}\mbox{}}
\fi

%%% Use protect on footnotes to avoid problems with footnotes in titles
\let\rmarkdownfootnote\footnote%
\def\footnote{\protect\rmarkdownfootnote}

%%% Change title format to be more compact
\usepackage{titling}

% Create subtitle command for use in maketitle
\providecommand{\subtitle}[1]{
  \posttitle{
    \begin{center}\large#1\end{center}
    }
}

\setlength{\droptitle}{-2em}

  \title{Práctica 2: Limpieza y análisis de datos}
    \pretitle{\vspace{\droptitle}\centering\huge}
  \posttitle{\par}
    \author{Raquel Gómez Pérez y Jorge Serra Planelles}
    \preauthor{\centering\large\emph}
  \postauthor{\par}
      \predate{\centering\large\emph}
  \postdate{\par}
    \date{9 de junio de 2020}

\usepackage[spanish]{babel}

\begin{document}
\maketitle

\hypertarget{resolucion}{%
\section{Resolución}\label{resolucion}}

\hypertarget{descripcion}{%
\subsection{1. Descripción del dataset}\label{descripcion}}
¿Por qué es importante y qué pregunta/problema pretende responder?.

	
\hypertarget{seleccion}{%
\subsection{2. Integración y selección de los datos de interés a analizar}\label{seleccion}}


\begin{itemize}
\tightlist
\item
  \textbf.\\  
\item
  \textbf.\\
\item
  \textbf.\\  
\end{itemize}

\newpage

\hypertarget{limpieza}{%
\subsection{3. Limpieza de los datos}\label{limpieza}} 

\hypertarget{ceros}{%
\subsubsection{3.1 Elementos vacíos}\label{ceros}}
¿Los datos contienen ceros o elementos vacíos? ¿Cómo gestionarías cada uno de estos casos?. 

\hypertarget{extremos}{%
\subsubsection{3.2 Valores extremos}\label{extremos}}
Identificación y tratamiento de valores extremos.
\hypertarget{analisis}{%
\subsection{4. Análisis de los datos}\label{analisis}}

\hypertarget{seleccion}{%
\subsubsection{4.1 Selección del grupo de datos}\label{seleccion}}
Selección de los grupos de datos que se quieren analizar/comparar (planificación de los análisis a aplicar). 
\hypertarget{normalidad}{%
\subsubsection{4.2  Normalidad y homogeneidad de la varianza}\label{normalidad}}
Comprobación de la normalidad y homogeneidad de la varianza. 

\hypertarget{estadistica}{%
\subsubsection{4.3  Pruebas estadísticas para comparar los grupos de datos}\label{estadistica}}
Aplicación de pruebas estadísticas para comparar los grupos de datos. En función de los datos y el objetivo del estudio, aplicar pruebas de contraste de hipótesis, correlaciones, regresiones, etc. Aplicar al menos tres métodos de análisis diferentes. 

\hypertarget{representacion}{%
\subsection{5. Representación de los resultados}\label{representacion}}
Representación de los resultados a partir de tablas y gráficas. 
\hypertarget{resolucion}{%
\subsection{6. Resolución del problema}\label{resolucion}}
Resolución del problema. A partir de los resultados obtenidos, ¿cuáles son las conclusiones? ¿Los resultados permiten responder al problema? 
\hypertarget{codigo}{%
\subsection{7. Código}\label{codigo}}

Tanto el código fuente escrito para la extracción de datos como el
dataset generado pueden ser accedidos a través de
\href{https://github.com/raquel8893/Tipologia-PRA2}{este enlace}.

\hypertarget{recursos}{%
\section{Recursos}\label{recursos}}

\begin{enumerate}
\def\labelenumi{\arabic{enumi}.}
\tightlist
\item
.\\
\item
.\\
\item
.\\
\item
.\\
\item
.\\

\end{enumerate}

\hypertarget{contribuciones}{%
\section{Contribuciones al trabajo}\label{contribuciones}}
\begin{tabular}{| c | c |}
\hline
Contribuciones & Firma \\ \hline
Investigación previa & RGP, JSP \\
Redacción de las respuestas & RGP, JSP \\
Desarrollo código & RGP, JSP \\ \hline
\end{tabular}

\end{document}
